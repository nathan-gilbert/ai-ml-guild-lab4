\documentclass[letterpaper,12pt]{article}
\usepackage{amssymb,latexsym,amsmath}
\usepackage{wasysym}
\usepackage{graphicx}
\usepackage{hyperref}
\hypersetup{
    colorlinks=true,
    linkcolor=blue,
    filecolor=magenta,
    urlcolor=cyan,
}

\begin{document}

\begin{flushleft}
Dimensionality Reduction with PCA\\
AI/ML Guild Lab \#4\\
\today \\
\end{flushleft}
Labeled data to create models for assigning labels to unseen data. This
lab is the first step in using unsupervised techniques. We're looking for
charateristics or patterns that inherently arise from the data itself.

Sometimes your data isn't very nice to you. The dimensions are just too damn
high! It's hard to vizualize, there are too many features that are too
interconnected. Principal Component Analysis is a technique that can help
resolve theses issues and can aid in model building.

\section{Principle Component Analysis}

\end{document}

